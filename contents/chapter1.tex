% !TEX root = ../thesis.tex
\chapter{前\hspace*{1\ccwd}言}
\chapteren{Introduction}

\section{声明}
\sectionen{Declaration}

\sysuthesis{}\footnote{\copyright~\number\year~\href{https://github.com/yanghw8}{yanghw8}\hspace*{2\ccwd}更新时间:\today}是旨在为中山大学熟悉\LaTeX{}语言的研究生提供一个方便易用的学位论文写作模版,其设置的排版格式力求尽可能符合《\href{https://graduate.sysu.edu.cn/sites/graduate.prod.dpcms4.sysu.edu.cn/files/2019-04/%E4%B8%AD%E5%B1%B1%E5%A4%A7%E5%AD%A6%E7%A0%94%E7%A9%B6%E7%94%9F%E5%AD%A6%E4%BD%8D%E8%AE%BA%E6%96%87%E6%A0%BC%E5%BC%8F%E8%A6%81%E6%B1%82.pdf}{中山大学研究生学位论文格式要求}》。首先声明:\strong{本模版不是官方模版,无法保证它完全符合学校的相关要求,在开始使用前,您同意,任何由于本模板而引起的论文格式审查问题与本模板作者无关。}

本模版暂时没有为本科生学位论文设置格式,如果您是本科生,请移步至\href{https://github.com/SYSU-SCC/sysu-thesis}{本科生模版}。如果您没有接触过\LaTeX{},又不打算花费时间和精力来入门,推荐您使用 Microsoft Office 套装来编写您的学位论文。如果您是\LaTeX{}语言的初学者,那么希望以下内容会对您的学习有所帮助。

\section{注意事项}
\sectionen{Notice}

本模版预设的封面、原创性声明及使用授权说明页、摘要页均以物理与天文学院的格式要求为主。如果您所在学院的要求与本模版预设的不同,建议参考以下 \textdagger 项的说明,正文以及参考文献部分各学院的要求应该是一致的。

\begin{itemize}
    \item 一般而言,通常不需要在中英文之间添加额外的空格,但为了代码的可读性(良好的习惯),还是建议在中文字符和 English 字符之间加上空格。
    \item[\textdagger] 对于扉页,如果对本模版预设的扉页不满意,可以使用 \texttt{pdfpages} 宏包中的 \texttt{\char92 includepdf} 命令导入您的扉页的 PDF 文件,例如
\begin{lstlisting}[language=TeX]
% \maketitle
\includepdf{titlepage.pdf}
\end{lstlisting}
    其中 \texttt{titlepage.pdf} 为扉页的 PDF 文件。
    \item[\textdagger] 同样的,对于原创性及使用授权说明页,也可以利用类似的方法:
\begin{lstlisting}[language=TeX]
% \makecopyright
\includepdf{copyrightpage.pdf}
\end{lstlisting}
    其中 \texttt{copyrightpage.pdf} 为扉页的 PDF 文件(可以为签字过后的扫描文件)。
    \item[\textdagger] 对于摘要页,在使用类似上述的命令之后,此外还应将摘要加入目录,因此建议使用以下命令
\begin{lstlisting}[language=TeX]
% % !TEX root = ../thesis.tex
% 中英文摘要
\begin{abszh}
    % \markboth{}{} % 如果不需要页眉显示摘要,取消注释掉此行
    摘要概括论文的主要信息,包括研究目的、方法、成果及最终结论。硕士论文摘要一般不超过1200字。博士论文摘要一般不超过2000字。关键词是供检索用的主题词条,应采用能覆盖论文主要内容的通用词。关键词一般列3~5个。

    摘要概括论文的主要信息,包括研究目的、方法、成果及最终结论。硕士论文摘要一般不超过1200字。博士论文摘要一般不超过2000字。关键词是供检索用的主题词条,应采用能覆盖论文主要内容的通用词。关键词一般列3~5个。
\end{abszh}

% 英文摘要
\begin{absen}
    % \markboth{}{} % 如果不需要页眉显示 Abstract,取消注释掉此行
    The first paragraph: The abstract summarizes the main information of the paper, including the purpose, methodology, results and final conclusions of the study. The abstract of the master's thesis generally does not exceed 1200 words. The abstract of the PhD thesis generally does not exceed 2000 words. Keywords are the subject terms for searching, and generic words that cover the main content of the paper should be used. Keywords generally list 3 to 5.
    
    The second paragraph: The abstract summarizes the main information of the paper, including the purpose, methodology, results and final conclusions of the study. The abstract of the master's thesis generally does not exceed 1200 words. The abstract of the PhD thesis generally does not exceed 2000 words. Keywords are the subject terms for searching, and generic words that cover the main content of the paper should be used. Keywords generally list 3 to 5.
\end{absen}
\addcontentsline{toc}{chapter}{\protect 摘\hspace*{2\ccwd}要}
\addcontentsline{toe}{chapter}{Abstract (In Chinese)}
\includepdf{abstract-zh.pdf}
\addcontentsline{toc}{chapter}{ABSTRACT}
\addcontentsline{toe}{chapter}{Abstract (In English)}
\includepdf{abstract-en.pdf}
\end{lstlisting}
    其中 \texttt{abstract-zh.pdf} 和 \texttt{abstract-en.pdf} 分别为中英文摘要页的 PDF 文件。
    \item 对于插图和表格的标题,本模版推荐使用 \texttt{bicaption} 宏包的 \texttt{\char92 bicaption} 命令,具体用法为:
\begin{lstlisting}[language=TeX]
\bicaption[中文短标题]{中文标题}[英文短标题]{英文标题}
\end{lstlisting}
    其中,短标题在插图索引或者表格索引中展示,而标题则在插图下方或者表格上方展示,见\autoref{fig:hexbin}示例。
    \item 在图表标题中,出现了引用文献后字号变回正文字号的问题,该问题有一个简单的解决方法,即使用 \texttt{\{\char92 cite\{key\}\}} 来避免上述问题发生。\strong{在弃用 \texttt{cite} 宏包之后,该问题似乎已经解决了。}
\end{itemize}

\section{使用说明}
\sectionen{Readme}

硕士论文不要求中英双目录和图表双标题,使用时仅使用
\begin{itemize}
    \item \texttt{\char92 chapter\{\}}
    \item \texttt{\char92 section\{\}}
    \item \texttt{\char92 subsection\{\}}
    \item \texttt{\char92 caption\{\}}
\end{itemize}
即可。博士使用
\begin{itemize}
    \item \texttt{\char92 chapter\{\}}、 \texttt{\char92 chapteren\{\}}
    \item \texttt{\char92 section\{\}}、 \texttt{\char92 sectionen\{\}}
    \item \texttt{\char92 subsection\{\}}、 \texttt{\char92 subsectionen\{\}}
    \item \texttt{\char92 bicaption\{\}\{\}}
\end{itemize}
以显示双语效果。此外,如果需要展示目录和索引,请直接使用\autoref{tab:listof}中的命令来打开。
\begin{table}[!htp]
    \zihao{5}
    \bicaption{目录索引}
    {Table of contents and  List of X}
    \label{tab:listof}
    \centering
    \begin{tabular}{ll}
        \toprule
        索引    & 命令 \\
        \midrule
        中文目录    &  \texttt{\char92 tableofcontents} \\
        英文目录    &  \texttt{\char92 tableofcontentsen} \\
        中文插图索引    &  \texttt{\char92 listoffigures} \\
        英文插图索引    &  \texttt{\char92 listoffiguresen} \\
        中文表格索引   &  \texttt{\char92 listoftables} \\
        英文表格索引   &  \texttt{\char92 listoftablesen} \\
        算法索引   &  \texttt{\char92 listofalgorithms} \\
        中文代码索引   &  \texttt{\char92 lstlistoflistings} \\
        英文代码索引   &  \texttt{\char92 listoflstlistingsen} \\
        \bottomrule
    \end{tabular}
\end{table}

\section{模版文件结构}
\sectionen{File structure}

本模版仅支持\hologo{XeTeX}排版引擎,其相应的编译命令称为 \texttt{xelatex},字符编码仅支持UTF-8,进行编译时,您需要使用正确编译器。本模版需要编译的主文件为 \texttt{thesis.tex},在编译时请选择 \texttt{xelatex}编译命令,由于是中文文档并且与\hologo{BibTeX}配合使用,请遵从以下编译步骤:

\begin{itemize}
    \item \texttt{xelatex}:生成 \texttt{.aux} 文件,里面包含了文档的结构信息和所有的内部引用(包括参考文献的引用);
    \item \texttt{bibtex}:\hologo{BibTeX} 读取 \texttt{.aux} 文件,根据给定的 \texttt{.bbl}  文件中指定的参考文献条目,生成 \texttt{.bbl} 文件,为格式化的参考文献列表;
    \item \texttt{xelatex}:将 \texttt{.bbl} 文件的参考文献列表嵌入到文档当中;
    \item \texttt{xelatex}:确保文档中的引用和编号与参考文献列表之间的对应关系是正确的,确保文档中的交叉引用(例如章节、图表、公式等)无误。
\end{itemize}
如此您将得到一个最终输出的正确的、完整的 PDF 文件。

\begin{figure}[p]
    \centering
    \begin{forest}
        directory,
        [sysuthesis-unofficial,  label=right:{\zihao{5}主目录}
          [codes, label=right:{\zihao{5}代码文件夹}
            [cohere.py, label=right:{\zihao{5}代码环境示例Python代码}]
            [vsc\_config.json, label=right:{\zihao{5}VSCode配置文件}]
          ]
          [contents, label=right:{\zihao{5}主要\TeX{}文件的文件夹}
            [abstract.tex, label=right:{\zihao{5}摘要内容编辑}]
            [ack.tex, label=right:{\zihao{5}后记内容编辑}]
            [appendix.tex, label=right:{\zihao{5}附录内容编辑}]
            [chapter1.tex,  label=right:{\zihao{5}章节内容编辑,下同}]
            [chapter2.tex]
            [chapter3.tex]
            [conclusion.tex, label=right:{\zihao{5}结论内容编辑}]
          ]
          [figures, label=right:{\zihao{5}图片文件夹}
            [hexbin.pdf]
            [histogram.pdf]
            [piechart.pdf]
          ]
          [.gitignore, label=right:{\zihao{5}git管理需要无视的文件}]
          [count\_chinese.py, label=right:{\zihao{5}中文字数统计Python代码}]
          [proposal.pdf, label=right:{\zihao{5}编译生成的开题报告PDF文件}]
          [proposal.tex, label=right:\strong{\zihao{5}需要编译的开题报告\TeX{}文件}]
          [README.md]
          [refs.bib, label=right:{\zihao{5}文件的语法格式应为\hologo{BibTeX}格式}]
          [setup.tex, label=right:{\zihao{5}配置论文信息、设置新命令以及调用宏包的文件}]
          [sysuthesis.cls, label=right:{\zihao{5}设置论文排版格式的类文档}]
          [thesis.pdf, label=right:{\zihao{5}编译生成的论文PDF文件}]
          [thesis.tex, label=right:\strong{\zihao{5}需要编译的论文主\TeX{}文件}]
        ]
    \end{forest}
    \bicaption{本模版的文件目录}
    {Directory structure of this template}
    \label{fig:dir}
\end{figure}

本模版的文件目录结构见\autoref{fig:dir}。重要的文件有:
\begin{itemize}
    \item \texttt{sysuthesis.cls}
\end{itemize}
分别是设置了本模版的论文排版格式和参考文献引用格式,在使用时,请您不要轻易修改该文件。您可以编辑的文件有:
\begin{itemize}
    \item \texttt{setup.tex}文件:编辑您的论文题目、作者姓名、专业、指导教师、关键词和学院及日期等关键信息,设置用于本文档的新\LaTeX{}命令以及调用的宏包;
    \item \texttt{contents}文件夹里面的文件:将您的文章内容分为多个部分编辑好,并在 \texttt{thesis.tex}中导入并排好顺序;
    \item \texttt{refs.bib}文件:用于编辑您的引用文献信息,请遵从\hologo{BibTeX}的语法格式,以免达成意料之外的效果;
    \item 为了方便起见,请将您的要使用的图片和代码文件放到相应的文件夹,以免造成不必要的混乱。
\end{itemize}

\section{\TeX{} Live套装及其他软件}
\sectionen{\TeX{} Live suite and other softwares}

\TeX{} Live是由国际\TeX{}用户组(\TeX{} Users Group,TUG)整理和发布的\TeX{}软件套装,包含与\TeX{}系统相关的各种程序、编辑与查看工具、常用宏包及文档、常用字体及多国语言支持。

\subsection{软件下载及安装}
\subsectionen{Download and install}

\TeX{} Live支持大家主要使用的Unix/Linux、Windows以及Mac OS等操作系统,它保持着每年一版的更新频率,是开源软件。可以直接到\href{https://www.tug.org}{\strong{TUG}}官网下载\href{https://www.tug.org/texlive}{\TeX{} Live},但可能受国内防火墙限制了下载速度,推荐大家到\href{https://mirrors.tuna.tsinghua.edu.cn/CTAN/}{清华大学开源软件镜像站}下载。请注意,对于Mac OS 系统,请选择下载\strong{Mac\TeX{}}。下载完成后,请根据提示进行安装,一般都是一路默认安装。

\subsection{\LaTeX{}编辑器}
\subsectionen{\LaTeX{} editor}

\LaTeX{}编辑器一般都会随着套件一起安装下来,如果你觉得默认的编辑器用起来不方便,下面推荐几个\LaTeX{}编辑器。
\begin{itemize}
    \item Visual Studio Code:这是一款由微软开发且跨平台的免费源代码编辑器。该软件支持语法高亮、代码自动补全、代码重构功能,默认支持非常多的编程语言。而且有内置的扩展程序商店,可以下载扩展支持你所需要的语言插件,\strong{需要配置环境}。请到\url{https://code.visualstudio.com}下载。
    \item Overleaf:这是一款\strong{在线协作}的\LaTeX{}编辑器,与很多科学杂志出版商有合作关系,上面不但提供官方期刊的\LaTeX{}模板,还能直接将文件提交至这些出版社。官方网站为\url{https://www.overleaf.com}。
    \item TeXstudio:这是一款开源的跨平台\LaTeX{}编辑软件,支持交互式拼写检查、代码折叠、语法高亮等特性。官网网站为\url{http://texstudio.sourceforge.net}。
\end{itemize}

\subsubsection{相关配置}

各种\LaTeX{}编辑器的配置可以轻易在网上找到,而且有的都比较简单。下面只介绍Visual Studio Code的配置。
\begin{itemize}
    \item 在扩展商店里找到\strong{LaTeX Workshop}插件,点击安装;
    \item 找到扩展设置(Extension Settings),找到 \texttt{settings.json}文件,编辑它,在里面加入你的配置代码。如\autoref{vscodeconfig}为我的配置;
    \item 之后可以在\TeX{}窗口里,选择对应的\strong{Build LaTeX project}进行编译。
\end{itemize}
\lstinputlisting[language=Java,
    caption={[VSCode \LaTeX{} 配置代码]Visual Studio Code \LaTeX{} 配置代码},
    caption2={[VSCode \LaTeX{} configuration]Visual Studio Code \LaTeX{} configuration},
    label=vscodeconfig]
{codes/vsc_config.json}

\section{推荐读物}
\sectionen{Further reading}

本文档不是\LaTeX{}的入门教程,因此不会对复杂的\LaTeX{}代码进行介绍。如果您只是用来编写您的学位论文,完全可以将源代码里的内容替换成你的内容,然后经过若干次复制、粘贴和修改,最终您会得到你所需要的文档。然而,有时候您想实现一些自己的个性化内容,希望下面推荐的读物可以帮助到您:
\begin{itemize}
    \item \href{https://www.overleaf.com/learn}{Overleaf:Documentation},在线英文文档,在里面实现不同功能的\LaTeX{}示例应有尽有;
    \item \href{http://www.ptep-online.com/ctan/lshort_chinese.pdf}{《一份不太简短的\hologo{LaTeX2e}介绍》};
    \item \href{https://github.com/wklchris/Note-by-LaTeX}{《简单粗暴\LaTeX{}》};
    \item 刘海洋:《\LaTeX{}入门》\cite{Liu:2013latexrm}。
\end{itemize}
最后祝您使用愉快!
% !TEX root = ../main.tex
\chapter{前言}

\blfootnote{\copyright\ \number\year\ \href{https://github.com/yanghw8}{yanghw8}}
\blfootnote{更新时间:\today}

\sysuthesis{}是旨在为中山大学熟悉\LaTeX{}语言的研究生提供一个方便易用的学位论文写作模版,其设置的排版格式力求尽可能符合《\href{https://sysgraduate.sysu.edu.cn/sites/graduate.prod.dpcms4.sysu.edu.cn/files/2019-04/%E4%B8%AD%E5%B1%B1%E5%A4%A7%E5%AD%A6%E7%A0%94%E7%A9%B6%E7%94%9F%E5%AD%A6%E4%BD%8D%E8%AE%BA%E6%96%87%E6%A0%BC%E5%BC%8F%E8%A6%81%E6%B1%82.pdf}{中山大学研究生学位论文格式要求}》。首先声明:\strong{本模版不是官方模版,无法保证它完全符合学校的相关要求,在开始使用前,您同意,任何由于本模板而引起的论文格式审查问题与本模板作者无关。}

本模版暂时没有为本科生学位论文设置格式,如果您是本科生,请移步至\href{https://github.com/SYSU-SCC/sysu-thesis}{本科生模版}。如果您没有接触过\LaTeX{},又不打算花费时间和精力来入门,推荐您使用Microsoft Office套装来编写您的学位论文。如果您是\LaTeX{}语言的初学者,那么希望以下内容会对您的学习有所帮助。

\section{模版文件结构}

本模版仅支持\hologo{XeTeX}排版引擎,其相应的编译器称为“xelatex”,字符编码仅支持UTF-8,进行编译时,您需要使用正确编译器。本模版需要编译的主文件为\texttt{main.tex},在编译时请选择“xelatex”编译器,由于是中文文档并且与\hologo{BibTeX}配合使用,请遵从以下编译步骤:
\begin{itemize}
    \item xelatex
    \item bibtex
    \item xelatex
    \item xelatex
\end{itemize}
如此您将得到一个最终完成的PDF文件。

本模版的文件目录结构见\ref{fig:dir}。重要的文件为\texttt{sysuthesis.cls},\texttt{sysuthesis.bst},分别是设置了本模版的论文排版格式和参考文献引用格式,在使用时,请您不要轻易修改这两个文件。您可以编辑的文件有:
\begin{itemize}
    \item \texttt{setup.tex}文件:编辑您的论文题目、作者姓名、专业、指导教师、关键词和学院及日期等关键信息,设置用于本文档的新\LaTeX{}命令以及调用的宏包;
    \item \texttt{contents}文件夹里面的文件:将您的文章内容分为多个部分编辑好,并在\texttt{main.tex}中导入并排好顺序;
    \item \texttt{ref}文件夹里的\texttt{refs.bib}文件:用于编辑您的引用文献信息,请遵从hologo{BibTeX}的语法格式,以免达成意料之外的效果;
    \item 为了方便起见,请将您的要使用的图片和代码文件放到相应的文件夹,以免造成不必要的混乱。
\end{itemize}

\begin{figure}[!htp]
    \centering
    \begin{forest}
        directory,
        [sysuthesis-unoffical,  label=right:{\zihao{5}主目录}
          [codes, label=right:{\zihao{5}代码文件夹}
            [cohere.py, label=right:{\zihao{5}代码环境示例Python代码}]
            [demo.ipynb, label=right:{\zihao{5}生成\texttt{figures}文件夹中图片的Python代码}]
          ]
          [contents, label=right:{\zihao{5}主要\TeX{}文件的文件夹}
            [abstract.tex, label=right:{\zihao{5}摘要内容编辑}]
            [ack.tex, label=right:{\zihao{5}后记内容编辑}]
            [appendix.tex, label=right:{\zihao{5}附录内容编辑}]
            [chapter1.tex]
            [chapter2.tex]
            [chapter3.tex]
            [summary.tex, label=right:{\zihao{5}结语内容编辑}]
          ]
          [figures, label=right:{\zihao{5}图片文件夹}
            [hexbin.pdf]
            [histogram.pdf]
            [piechart.pdf]
            [sysu.pdf]
          ]
          [ref, label=right:{\zihao{5}放置引用文献信息的文件的文件夹}
            [refs.bib, label=right:{\zihao{5}文件的语法格式应为\hologo{BibTeX}格式}]
          ]
          [main.tex, label=right:\strong{\zihao{5}需要编译的主\TeX{}文件}]
          [main.pdf, label=right:{\zihao{5}编译生成的PDF文件}]
          [README.md]
          [setup.tex, label=right:{\zihao{5}配置论文信息、设置新命令以及调用宏包的文件}]
          [sysuthesis.bst, label=right:{\zihao{5}设置参考文献格式的文件}]
          [sysuthesis.cls, label=right:{\zihao{5}设置论文排版格式的类文档}]
        ]
    \end{forest}
    \caption{本模版的文件目录}
    \label{fig:dir}
\end{figure}

\section{\TeX{} Live套装及其他软件}

\TeX{} Live是由国际\TeX{}用户组(\TeX{} Users Group,TUG)整理和发布的\TeX{}软件套装,包含与\TeX{}系统相关的各种程序、编辑与查看工具、常用宏包及文档、常用字体及多国语言支持。

\subsection{软件下载及安装}

\TeX{} Live支持大家主要使用的Unix/Linux、Windows以及Mac OS等操作系统,它保持着每年一版的更新频率,是开源软件。可以直接到\href{https://www.tug.org}{\strong{TUG}}官网下载\href{https://www.tug.org/texlive}{\TeX{} Live},但可能受国内防火墙限制了下载速度,推荐大家到\href{https://mirrors.tuna.tsinghua.edu.cn/CTAN/}{清华大学开源软件镜像站}下载。请注意,对于Mac OS 系统,请选择下载\strong{Mac\TeX{}}。下载完成后,请根据提示进行安装,一般都是一路默认安装。

\subsection{\LaTeX{}编辑器}

\LaTeX{}编辑器一般都会随着套件一起安装下来,如果你觉得默认的编辑器用起来不方便,下面推荐几个\LaTeX{}编辑器。

\begin{itemize}
    \item Visual Studio Code:这是一款由微软开发且跨平台的免费源代码编辑器。该软件支持语法高亮、代码自动补全、代码重构功能,默认支持非常多的编程语言。而且有内置的扩展程序商店,可以下载扩展支持你所需要的语言插件,\strong{需要配置环境}。请到\url{https://code.visualstudio.com}下载。
    \item Overleaf:这是一款\strong{在线协作}的\LaTeX{}编辑器,与很大科学杂志出版商有合作关系,上面不但提供官方期刊的\LaTeX{}模板,还能直接将文件提交至这些出版社。官方网站为\url{https://www.overleaf.com}。
    \item TeXstudio:这是一款开源的跨平台\LaTeX{}编辑软件,支持交互式拼写检查、代码折叠、语法高亮等特性。官网网站为\url{http://texstudio.sourceforge.net}。
\end{itemize}

\subsubsection{相关配置}

各种\LaTeX{}编辑器的配置可以轻易在网上找到,而且有的都比较简单。下面只介绍Visual Studio Code的配置。
\begin{itemize}
    \item 在扩展商店里找到\strong{LaTeX Workshop}插件,点击安装;
    \item 找到扩展设置(Extension Settings),找到\texttt{settings.json}文件,编辑它,在里面加入你的配置代码。以下是我的配置:
\begin{lstlisting}
    "latex-workshop.latex.tools": [
        {
            "name": "xelatex",
            "command": "xelatex",
            "args": [
                "-synctex=1",
                "-interaction=nonstopmode",
                "-file-line-error",
                "%DOCFILE%"
            ]
        },
        {
            "name": "pdflatex",
            "command": "pdflatex",
            "args": [
                "-synctex=1",
                "-interaction=nonstopmode",
                "-file-line-error",
                "%DOCFILE%"
            ]
        },
        {
            "name": "latexmk",
            "command": "latexmk",
            "args": [
                "-synctex=1",
                "-interaction=nonstopmode",
                "-file-line-error",
                "-pdf",
                "-outdir=%OUTDIR%",
                "%DOCFILE%"
            ]
        },
        {
            "name": "bibtex",
            "command": "bibtex",
            "args": [
                "%DOCFILE%"
            ]
        }
    ],
    "latex-workshop.latex.recipes": [
        {
            "name": "XeLaTeX",
            "tools": [
                "xelatex"
            ]
        },
        {
            "name": "PDFLaTeX",
            "tools": [
                "pdflatex"
            ]
        },
        {
            "name": "BibTeX",
            "tools": [
                "bibtex"
            ]
        },
        {
            "name": "LaTeXmk",
            "tools": [
                "latexmk"
            ]
        },
        {
            "name": "xelatex -> bibtex -> xelatex*2",
            "tools": [
                "xelatex",
                "bibtex",
                "xelatex",
                "xelatex"
            ]
        },
        {
            "name": "pdflatex -> bibtex -> pdflatex*2",
            "tools": [
                "pdflatex",
                "bibtex",
                "pdflatex",
                "pdflatex"
            ]
        }
    ],
    "latex-workshop.view.pdf.viewer": "tab",
    "latex-workshop.bind.altKeymap.enabled": true,    
\end{lstlisting}
    \item 之后可以在\TeX{}环境里,选择对应的\strong{Build LaTeX project}进行编译。
\end{itemize}

\section{推荐读物}

本文档不是\LaTeX{}的入门教程,因此不会对复杂的\LaTeX{}代码进行介绍。如果您只是用来编写您的学位论文,完全可以将源代码里的内容替换成你的内容,然后经过若干次复制、粘贴和修改,最终您会得到你所需要的文档。然而,有时候您想实现一些自己的个性化内容,希望下面推荐的读物可以帮助到您:
\begin{itemize}
    \item \href{https://www.overleaf.com/learn}{Overleaf:Documentation},在线英文文档,在里面实现不同功能的\LaTeX{}示例应有尽有;
    \item \href{http://www.ptep-online.com/ctan/lshort_chinese.pdf}{《一份不太简短的\hologo{LaTeX2e}介绍》};
    \item \href{https://github.com/wklchris/Note-by-LaTeX}{《简单粗暴\LaTeX{}》};
    \item 刘海洋:《\LaTeX{}入门》\cite{Liu:2013latexrm}。
\end{itemize}
最后祝您使用愉快!
% !TEX root = ../thesis.tex
\chapter{一些重要的\LaTeX{}环境}\label{chap:evm}
\chapteren{Important \LaTeX{} environments}

本模版中的公式、插图、表格和章节等,均用\texttt{\textbackslash lable\{<key>\}}来在\LaTeX{}代码中标记位置,用\texttt{\textbackslash ref\{<key>\}}来在代码中引用,其中\texttt{<key>}为自定义的标签。

\section{公式示例}
\sectionen{Formula example}

文中的公式建议使用 \texttt{amsmath} 宏包的 \texttt{align} 环境,该环境在对多行公式对齐方面具有很大的优势,具体的讨论请看知乎用户\href{https://www.zhihu.com/people/bo-xue-duo-wen-63}{\strong{博闻多学}}的\href{https://www.zhihu.com/question/477805692/answer/2045084752}{\strong{回答}}。

下面进行公式示例。普通公式:
\begin{align}
    a+b=x.
\end{align}
带有积分和分隔的公式:
\begin{align}
   \int^{\infty}_{0} f(x)\dd{x}, \qquad \oint_{C} f(z)\dd {z}.
\end{align}
多行公式:
\begin{align}
    \left(1+x\right)^{\alpha} &= \sum^{\infty}_{n=0}\left(\begin{matrix} \alpha \\ n\end{matrix}\right)x^n \nonumber \\ 
    &= 1 + \alpha x + \frac{\alpha(\alpha-1)}{2!}x^2 + \cdots + \frac{\alpha(\alpha-1)\cdots(\alpha-n+1)}{n!} + \cdots
    \label{eqn:taylorseries}
\end{align}
这里注意,对不需要编号的行要取消公式编号,即要在该行公式的源代码后边使用\texttt{\textbackslash nonumber}命令。

公式的引用示例:\ref{eqn:taylorseries}为泰勒级数。

\section{插图示例}
\sectionen{Figure example}

文中插图的插图建议使用\texttt{graphicx}宏包的\texttt{figure}环境搭配\texttt{\textbackslash includegraphics}命令。例如:
\begin{figure}[htbp]
	\centering
	\includegraphics[width=1\textwidth]{figures/hexbin.pdf}
	\bicaption[六边形分bin图]{六边形分bin图六边形分bin图六边形分bin图}[Hexagonal binned plot]
    {Hexagonal binned plot Hexagonal binned plot Hexagonal binned plot}
	\label{fig:hexbin}
\end{figure}
\begin{figure}[htbp]
	\centering
    \begin{subfigure}{0.45\textwidth}
        \centering
	    \includegraphics[width=1\textwidth]{figures/histogram.pdf}
	    \bicaption{柱状图}
        {Histogram}
    \end{subfigure}
    \begin{subfigure}{0.45\textwidth}
        \centering
	    \includegraphics[width=1\textwidth]{figures/piechart.pdf}
	    \bicaption{饼状图}
        {Pie chart}
    \end{subfigure}
    \bicaption{子图示例}
    {Subfigure example}
    \label{fig:subfig}
\end{figure}

插图的引用示例:\ref{fig:hexbin}是普通插图。

\section{表格示例}
\sectionen{Table example}

文中的表格建议使用\texttt{table}环境里嵌套\texttt{tabular}环境。
\begin{table}[htbp]
    \zihao{5}
    \bicaption{2022年北京冬奥会奖牌榜}
    {2022 Beijing winter Olympics medals}
    \label{tab:01}
    \centering
    \begin{tabular}{ccrrrr}
        \toprule
        总排名 & 国家/地区 & 金牌 & 银牌 & 铜牌 & 合计  \\ 
        \midrule
        1 & 挪威 & 16 & 8 & 13 & 37\\
        2 & 德国 & 12 & 10 & 5 & 27\\
        3 & 中国 & 9 & 4 & 2 & 15\\
        4 & 美国 & 8 & 10 & 7 & 25\\
        5 & 瑞典 & 8 & 5 & 5 & 18\\
        6 & 荷兰 & 8 & 5 & 4 & 17\\
        7 & 奥地利 & 7 & 7 & 4 & 18\\
        8 & 法国 & 7 & 2 & 5 & 14\\
        9 & 俄罗斯奥林匹克委员会\tablefootnote{俄罗斯由于被禁赛,不能以国家名义参加奥运会,不能使用国旗和国歌。因此俄罗斯代表团绕过禁令,以俄罗斯奥委会(Russian Olympic Committee)的名义参赛,以俄罗斯奥委会的会旗作为代表团的团旗,以柴可夫斯基的《第一钢琴协奏曲》作为团歌\cite{ROC}。} 
            & 6 & 12 & 14 & 32\\ 
        10 & 法国 & 5 & 7 & 2 & 14\\
        \bottomrule
    \end{tabular}
\end{table}
这里需要注意,如果需要在表格内添加注释,请使用\texttt{tablefootnote}宏包的\texttt{\textbackslash tablefootnote}命令。如果要制作长表格,请使用\texttt{longtable}宏包的\texttt{longtable}环境。

此外,在科技论文的排版中,一般使用三线表。推荐您使用 \texttt{booktabs} 宏包,该宏包支持三线表。本模版已经装载了 \texttt{booktabs} 宏包,可使用 \texttt{\textbackslash toprule}、\texttt{\textbackslash midrule} 和 \texttt{\textbackslash bottomrule} 命令替换掉对应的 \texttt{\textbackslash hline} 即可。

表格的引用示例:\ref{tab:01}是2022年北京冬奥会奖牌榜。

\section{其他数学环境示例}
\sectionen{Other theorem environments example}

以下是本模版预设的数学环境示例:

\begin{assumption}[连续统假设]
    不存在一个基数绝对大于可数集而绝对小于实数集的集合。
\end{assumption}
\begin{axiom}[平行公理]
    若两条直线都与第三条直线相交,并且在同一边的内角之和小于两个直角,则这两条直线在这一边必定相交。
\end{axiom}
\begin{conjecture}[黎曼猜想]
    黎曼$\zeta$函数
    \begin{align}
        \zeta(s) = \frac{1}{1^s} + \frac{1}{2^s} + \frac{1}{3^s} + \frac{1}{4^s} + \cdots
    \end{align}
    非平凡点的实数部分是$\frac{1}{2}$。
\end{conjecture}
\begin{definition}[定义的定义]
    对一个概念或者词或者词组的定义是描写其内涵,即描写其所有和仅有的元素的共有特征。其外延是所有这个概念、词或者词组包含的事物。
\end{definition}
\begin{example}
    举个栗子。
\end{example}
\begin{exercise}
    TiMi,发出学习的声音。
\end{exercise}
\begin{lemma}[欧几里得引理]
    如果一个正整数整除另外两个正整数的乘积,第一个整数与第二个整数互质,那么第一个整数整除第三个整数。
\end{lemma}
\begin{problem}
    花儿为什么这样红?
\end{problem}
\begin{proposition}
    通过一个不在直线上的点,有且仅有一条不与该直线相交的直线。
\end{proposition}
\begin{theorem}[诺特定理]
    对于每个局部作用下的可微对称性,存在一个对应的守恒流。另言之,每个连续对称性都有着相应的守恒定律。
\end{theorem}
\begin{corollary}
    推论往往在定理后出现。如果命题B能够被简单明了的从命题A推导出,则称B为A的推论。
\end{corollary}
\begin{solution}
    这个问题无解。
\end{solution}
\begin{proof}
    因为爱情,不会轻易悲伤,所以一切都是幸福的模样。
\end{proof}

\section{代码示例}
\sectionen{Code listings example}

在论文中插入代码,我们使用的是\texttt{listings}宏包的\texttt{lstlisting}环境,如:
\begin{lstlisting}[language=Python,
    caption1=Python 画图代码1,
    caption2=Python ploting code 1]
import numpy as np
import matplotlib.pyplot as plt

# Fixing random state for reproducibility
np.random.seed(19680801)

dt = 0.01
t = np.arange(0, 30, dt)
nse1 = np.random.randn(len(t))                 # white noise 1
nse2 = np.random.randn(len(t))                 # white noise 2
    
# Two signals with a coherent part at 10Hz and a random part
s1 = np.sin(2 * np.pi * 10 * t) + nse1
s2 = np.sin(2 * np.pi * 10 * t) + nse2

fig, axs = plt.subplots(2, 1)
axs[0].plot(t, s1, t, s2)
axs[0].set_xlim(0, 2)
axs[0].set_xlabel('time')
axs[0].set_ylabel('s1 and s2')
axs[0].grid(True)
 
cxy, f = axs[1].cohere(s1, s2, 256, 1. / dt)
axs[1].set_ylabel('coherence')

fig.tight_layout()
plt.show()
\end{lstlisting}
或\texttt{\textbackslash lstinputlisting}命令,如:
\lstinputlisting[language=Python,
caption1=Python 画图代码2,
caption2=Python ploting code 2]
{codes/cohere.py}
建议代码不设置\texttt{caption}选项,也不要使用\texttt{\textbackslash ref}来引用,因为我还没设置好 (> <)。

\section{参考文献}\label{sec:bibstyle}
\sectionen{References example}

\subsection{引用格式}
\subsectionen{Citation format}

例如:
\begin{enumerate}
    \item 这是一个期刊的引用\cite{LIGOScientific:2017zic};
    \item 这是一个图书的引用\cite{Rubakov:2017xzr,Zhang:2021};
    \item 这是一个研讨会论文的引用\cite{Tanikawa:2021+x};
    \item 这是博士论文的引用\cite{Migenda:2019xbm,HuangGuoYuan:2020},这是硕论文的引用\cite{Shojaeifar:2015csv,SongRen:2020};
    \item 这是电子文献的引用\cite{Piro:2021oaa,bilibili:read}。
    \item 这是报纸的引用\cite{Li:2005}。
\end{enumerate}

\subsection{参考文献引用说明}
\subsectionen{References citation notes}

参考文献的引用格式已经由\texttt{sysuthesis.bst}设置好了。引用时,请将引用信息编入\texttt{ref}文件夹的\texttt{refs.bib}文件中,语法要符合\hologo{BibTeX}格式,并在文中引用处使用\texttt{\textbackslash cite}命令。建议只使用\ref{tab:entrytypes}中的几种\hologo{BibTeX}条目类型(Entry Types)。
\begin{table}[htbp]
    \zihao{5}
    \bicaption{本模版设置好的\hologo{BibTeX}字段类型}
    {The entry types used in this template}
    \label{tab:entrytypes}
    \centering
    \begin{tabular}{lll}
    \toprule
    引用文献的类型  & 标识 & Entry Types \\
    \midrule
    期刊    &[J] & \texttt{article} \\
    图书    &[M] & \texttt{book} \\
    研讨会论文 &[C]    & \texttt{conference} \\  
    博士论文   &[D]  & \texttt{phdthesis} \\
    硕士论文   &[D]  & \texttt{mastersthesis} \\
    报纸      &[N] & \texttt{newspaper} \\
    电子公告   &[EB/OL] & \texttt{online} \\
    电子期刊   &[J/OL] & \texttt{articleonline} \\
    电子图书   &[M/OL] & \texttt{bookonline} \\
    \bottomrule
    \end{tabular}
\end{table}
请注意:\strong{如果引用的是中文文献,请再额外添加\texttt{language} 字段,并让它不为空,否者将输出英文引用格式}。例如,这是一个\texttt{article}条目类型的源代码:
\begin{lstlisting}[language=TeX,
    caption1={\hologo{BibTeX}代码示例},
    caption2={\hologo{BibTeX} code example}]
% ./ref/refs.bib
@article{ZhaoWen:2017twxjz,
    author={赵文 and 张星 and 刘小金 and 张杨 and 王运永 and 张帆 and 肇宇航 and 郭越凡 and 陈奕康 and 艾舜柯 and 朱宗宏 and WANG Xiao-ge and LEBIGOT Eric and 都志辉 and 曹军威 and 钱进 and 殷聪 and 王建波 and BLAIR David and JU Li and ZHAO Chun-nong and WEN Lin-qing},
    title={ 引力波与引力波源 },
    journal={天文学进展},
    language={中文},
    year={2017},
    volume={35},
    number={3},
    pages={316-344},
    month={1},
}
\end{lstlisting}
请点击\cite{ZhaoWen:2017twxjz}查看该期刊文章的引用效果。这是中文图书的引用效果\cite{Huang:2012hxwl}

\section{注释}
\sectionen{Remarks}

注释:可以用“脚注”或“文后注”来标注引用著作中的一些观点和案例,但全文标注方式应统一,本文统一使用“脚注”\footnote{这里是注释内容。}。

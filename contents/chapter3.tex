\chapter{更新描述}
\chapteren{Update Description}

\section*{\texttt{v1.1.3 2024/03/28}}
\begin{itemize}
    \item 撤销将 \texttt{\char92 ref} 命令的引用格式重设为 \texttt{(\char92 autoref\{key\})} 的更改。
    \item 解决了一些与 \texttt{hyperref} 宏包的冲突问题。
\end{itemize}

\section*{\texttt{v1.1.2 2024/03/14}}
\begin{itemize}
    \item 放弃自制的 \texttt{sysuthesis.bst},改用 \texttt{gbt7714} 宏包。
    \item 增加  \texttt{count\_chinese.py} Python 脚本,用于统计中文字数。
    \item 重新设置论文信息的设置方式,即键值对(key-value)的格式,更加友好。
    \item 修改了\texttt{checkmode}的版面,去除无效的空白页。
    \item 添加了中山大学的颜色 \texttt{sysugreen}、 \texttt{sysured} 和 \texttt{spablue}。
    \item 给出了长表格的示例,并配置了 \texttt{tabularray} 的风格。
\end{itemize}

\section*{\texttt{v1.1.1 2023/03/30}}
\begin{itemize}
    \item 使用 \texttt{\char92 raggedbottom} 调整页面的垂直对齐方式, 当页面内容不足时, 这将减少页面顶部和底部之间的间距,使得页面看起来更加紧凑。
    \item 增加 \texttt{fontset} 选项 (\texttt{<default> = fandol}),指定\CTeX{}宏集加载的字库,详情请查看\CTeX{}宏集的具体说明。例如,如果您的系统为Windows,则可以用以下选项:
\begin{lstlisting}[language=TeX]
\documentclass[doctype=thesis,printmode=final,openright,blankleft,fontset=windows]{sysuthesis}
\end{lstlisting}
    如果您在 Overleaf 上编译,则可以设置为:
\begin{lstlisting}[language=TeX]
\documentclass[doctype=thesis,printmode=final,openright,blankleft,fontset=ubuntu]{sysuthesis}
\end{lstlisting}
    目前 Mac OS 可以暂时使用 \texttt{fontset=macnew},依然解决不了找不到对应字体的警告问题,但无伤大雅。
    \item 对一些笔误进行了修改。
\end{itemize}

\section*{\texttt{v1.1.0 2023/03/03}}
\begin{itemize}
    \item 增加以下模版选项:
    \begin{itemize}
        \item \texttt{doctype},可选 \texttt{thesis}|\texttt{proposal} (\texttt{<default> = thesis}),分别为学位论文和开题报告的格式。
        \item \texttt{printmode},可选 \texttt{final}|\texttt{checkmode}|\texttt{blindmode} (\texttt{<default> = final}),分别为终稿、查重和盲审的打印模式。
        \item \texttt{openright}|\texttt{openany},互为 \texttt{true}|\texttt{false} (\texttt{<default> = openright})。\\ \texttt{openright} 选项为每一章在右页(奇数页)开始,\texttt{openright} 选项为在上一章结束的下一页开始。
        \item \texttt{blankleft} (\texttt{<default> = false}),当 \texttt{blankleft = true} 时,章节结束的偶数页如果没有内容,使之空白,但页码计数器仍然有效。
    \end{itemize}
    \item 增加了 \texttt{appendixenv}、\texttt{publications}和\texttt{achievements} 环境,分别为附录、学术论文发表列表和学术成果列表的环境。
    \item 对论文扉页进行了微调。
    \item 修改 \texttt{lstlisting} 双语标题格式,微调相关颜色。
    \item 增加了 NASA/ADS Export Citation 的期刊名命令,不需要再手动修改以避免 \hologo{BibTeX} 编译出错。
\end{itemize}

\section*{\texttt{v1.0.1 2022/03/06}}
\begin{itemize}
    \item 最新适配物理与天文学院的格式要求,调整了参考文献的引用格式并添加文献类型标识,将中文与西文之间的一个半角字符的自动间距关闭。\texttt{\char92 texttt}命令只在本文档用以展示命令,不建议大家使用。
\end{itemize}

\section*{\texttt{v1.0 2022/02/23}}
\begin{itemize}
    \item 最初版本。
\end{itemize}
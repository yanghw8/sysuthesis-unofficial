% !TEX encoding = UTF-8 Unicode
\documentclass[proposal]{sysuthesis}

% 设置论文的基本信息,包括题目、作者、专业、导师、学院、摘要和关键词等必要信息
% !TEX root = ./main.tex
% 设置论文题目,第一个“{}”为中文题目,第二个“{}“为英文题目
\title{中山大学研究生学位论文\LaTeX{}非官方模版}
{English Title for Sun Yat-sen University Thesis \LaTeX{} Unofficial Template}

% 设置作者,第一个“{}”为中文名字,第二个“{}“为英文名字
\author{张三}{Zhang San}

% 设置专业,第一个“{}”为中文专业名,第二个“{}“为英文专业名
\major{天体物理}{Astrophysics}

% 设置指导教师,第一个“{}”为中文名字,第二个“{}“为英文名字
\supervisor{李四}{Li Si}

% 设置学位名
\degree{硕士}

% 设置关键词,第一个“{}”为中文关键词,第二个“{}“为英文关键词
\keywords{天体物理、引力波、黑洞}{Astrophysics, Gravitational Wave, Black Holes}

% 设置学院
\school{物理与天文学院}

% 设置校区
\campus{珠海}

% 设置日期,默认是当天时间,具体到日
\date{2022年6月}

% 设置新的latex命令
\newcommand{\dd}{\mathrm{d}}
% 调用的宏包
\usepackage{longtable}
\usepackage{tablefootnote}
\usepackage{hologo}


\usepackage{multirow}

\begin{document}

% 前置部分
\frontmatter

% 扉页
\maketitle

% 目录
\tableofcontents

% 主体部分
\mainmatter

% 正文

\section*{说\hspace*{2\ccwd}明}

\begin{itemize}
    \item[一、]开题报告应包括下列主要内容,见目录。
    \item[二、]开题报告字数应不少于1.5万字。
    \item[三、]开题报告最迟应于第四学期结束前完成。
    \item[四、]若本次开题报告未通过,需在6-12个月内再次进行开题报告。第二次学位论文开题报告仍未通过者,按评议小组的建议进行后续分流工作。
    \item[五、]开题报告结束后,评议小组要填写《博士学位论文开题报告评议结果》,与主席签字的“博士学位论文开题答辩原始记录”一同上报学院研究生教学秘书备案。学生需将修改过的《开题报告》和《博士学位论文开题报告修改情况确认表》于开题答辩后一周内上交学院研究生教学秘书备案。
    \item[六、]字体、字号及其他规定
    
    论文中所用中文字体(除各级标题外)为宋体,各级标题用黑体;论文中所用数字、英文为新罗马字体。

    \begin{tabular}{ll}
        节标题         &小3号字,建议段前0.5行,段后0.5行;\\
        条标题         &4号字,建议段前0.5行,段后0.5行;\\
        款、项标题      &小4号字, 建议段前0行,段后0行;\\
        正文           &小4号字,建议段前0行,段后0行,每页约33行。
    \end{tabular}
    \item[七、]层次代号及说明 
    \begin{table}[h]
        \zihao{5}
        \centering
        \begin{tabular}{p{0.1\linewidth}|p{0.45\linewidth}|p{0.35\linewidth}}
            \hline 
            层次名称 & 示例 & 说明 \\
            \hline 
            节 &1~$\Box\Box\cdots\cdots\Box$ & \multirow{3}*{题序顶格书写,阐述内容另起一段} \\
            \cline{1-2}
            条 &1.1~$\Box\Box\cdots\cdots\Box$ & \\
            \cline{1-2}
            款 &1.1.1~$\Box\Box\cdots\cdots\Box$  & \\
            \hline
            项 &\hspace*{2\ccwd}(1)~$\Box\Box\cdots\cdots\Box$ \hspace*{2\ccwd}$\Box\Box\cdots\cdots\Box\Box$  & 题序空4个半角字符书写,内容空4个半角字符接排 \\
            \hline
        \end{tabular}
    \end{table}
    \item[八、]常用的四种参考文献类型标注形式。例如:
    \begin{enumerate}
        \item 这是一个期刊的引用\cite{LIGOScientific:2017zic};
        \item 这是一个图书的引用\cite{Rubakov:2017xzr,Zhang:2021};
        \item 这是一个研讨会论文的引用\cite{Tanikawa:2021+x};
        \item 这是博士论文的引用\cite{Migenda:2019xbm,HuangGuoYuan:2020},这是硕论文的引用\cite{Shojaeifar:2015csv,SongRen:2020};
        \item 这是电子文献的引用\cite{Piro:2021oaa,bilibili:read}。
        \item 这是报纸的引用\cite{Li:2005}。
    \end{enumerate}
\end{itemize}

    
\section{课题来源及研究的目的和意义}
\subsection{课题来源或研究背景}
\subsection{研究的目的及意义}
(不少于1000字)

\section{国内外在该方向的研究现状及分析}
(文献综述)
\subsection{国外研究现状}
\subsection{国内研究现状}
(注意对所引用国内外文献的准确标注)
\subsection{国内外文献综述的简析}
(不少于1000字)
(综合评述:国内外研究取得的成果,存在的不足或有待深入研究的问题)

\section{学位论文的主要研究内容、实施方案及其可行性论证前期研究与论证工作的结果}
\subsection{主要研究内容}
(不少于2000字)
(撰写宜使用将来时态,不能只列出论文目录来代替对研究内容的分析论述)
\subsection{实施方案及其可行性论证}
(不少于3000字)

\section{已完成的研究工作}
(详细撰写目前已进行的研究工作内容和完成情况)

\section{论文进度安排,预期达到的目标}
\subsection{进度安排}
(从确定博士选题收集文献写起)
\subsection{预期达到的目标}

\section{学位论文预期创新点}
(要根据研究内容和国内外研究现状准确提炼,充分体现创新性)

\section{为完成课题已具备和所需的条件、外协计划及经费}

\section{预计研究过程中可能遇到的困难、问题,以及解决的途径}

\section{主要参考文献}
(应在50篇以上,其中外文资料不少于二分之一,参考文献中近五年(从开题当年算起)内发表的文献一般不少于三分之一,且必须有近二年内发表的文献资料)。



\makebib{refs}

\cleardoublepage
\end{document}